\documentclass[11pt]{article}

% deutche Umlaute
\usepackage[utf8]{inputenc}
\usepackage[T1]{fontenc}

\usepackage[a4paper, left=2.5cm, right=2.5cm, top=3cm]{geometry}

%\usepackage[iso,german]{isodate}
\usepackage{fancyhdr}

\title{Gliederung}
\cfoot{hallo}

\begin{document}
\maketitle

\section{Einleitung}
\subsection{„Aktueller Stand der Forschung“}
\begin{itemize}
\item NGS
\item Massen an Daten, Größe der Dateien
\item Schwierigkeiten der Speicherung und Analyse
\end{itemize}

\subsection{Motivation/Ziel}
\begin{itemize}
\item Verringerung von Speicherplatz sowohl auf Dateiebene als auch (bzw. primär) für den Arbeitsspeicher
\item Funktionalität in und zur Benutzung von SeqAn (DeltaMap)
\end{itemize}

\subsection{Zusammenfassung aller anderen Bereiche}
\begin{itemize}
\item Wie wurde das Ziel umgesetzt?
\item Was kam heraus? 
\item Bewertung der Lösung
\item Was lässt sich noch Verbessern?
\end{itemize}

\section{Hintergrund}
\subsection{Speicheranalyse genomischer Daten}
\begin{itemize}
\item Wie sieht genomische DNA grob aus
\item Wie wird genomische DNA gespeichert (FASTA, VCF)
\item Beispiele
\item Ansprüche an einen Computer (bei bestimmten Algorithmen) bezogen auf Geschwindigkeit und vor allem Speicherplatz
\end{itemize}

\subsection{Lösungsansätze der Literatur: Referentielle Kompression}
\begin{itemize}
\item Kurz ansprechen was es sonst gibt: Lempel-Ziv...
\item Was ist referentielle Kompression?
\item (Was sind Delta-Events)
\item Welchen Ansätze wurden schon nachgegangen, welche Algorithmen gibt es bereits für Kompression von Dateien?
\item welche Algorithmen gibt es bereits die auf komprimierten Daten Algorithmen laufen lassen kann?
\item SeqAn (Delta Map)
\end{itemize}

\section{Methoden}
\begin{itemize}
\item Grundidee des Algorithmus'
\item Input, Output
\item Defitionen: Seed, SearchField, DeltaEvent, DependantRegion, JournaledString/Entrie, DeltaMap
\end{itemize}
\subsection{Variant Calling}
\subsubsection{Finde Seeds}
\begin{itemize}
\item Q-gram Index (1. Sequenz des Inputs, Openadressing())
\item Jede weitere Sequenz wird gegen den Index nach Hits durchsucht -> Hit = Seed
\item Problem: O(n) sehr langsam vor allem bei Repeats
\item Lösung für Repeats (Definition, Einschränkung)
\item (Lösung für Schnelligkeit im nächsten Absatz)
\end{itemize}

\subsubsection{Idee der Funktion (parallel)fastFirstSeeding()}
\begin{itemize}
\item Hits(Seeds) werden wenn sie gefunden werden erweitert (extend)
\item (Maximale Seed Größe, Grenze um Chaining Problem zu beschleunigen)[vielleicht erst bei Chaining-section]
\item Scoring dieser Seeds (nach Länge und Lage zur Hauptdiagonalen)
\item Offset = Länge des Seeds mit besten Score
\item Anmerkung: Parallelisierung durch Aufteilen in Abschnitte
\end{itemize}

\subsubsection{Chaining Algorithmus}
\begin{itemize}
\item Seeds werden zu einem aligniert
\item kurze Beschreibung des Algorithmus'
\item Beschleunigung durch Maximale Seed Größe
\end{itemize}

\subsubsection{Iterativer Schritt}
\begin{itemize}
\item fastFirstSeeding ist sehr grob
\item um trotzdem sensitivität zu erhalten wird ein iterativer Schritt durchgeführt
\item N's werden zugelassen was sich aber oft negativ auf die Laufzeit auswirkt -> ignorieren der N-qgramme beim seeding (sollte durch extendSeeds eh berücksichtigt werden)
\end{itemize}

\subsubsection{Überführen von Seeds in Delta Events}
\begin{itemize}
\item Kette von Seeds bilden ein Alignment
\item Alignment-Repräsentation mit Delta-Events
\item Überführung von Seeds in Delta-Events -> Manhattan Distanz
\end{itemize}

\subsection{Event Processing (on reference)}
\begin{itemize}
\item Prozessieren um zu komprimieren. Kurze Zusammenfassung:
\item Zwei verschiedene Ansätze:
\item 1. Kombinieren von gleichen Events -> "Multiples Alignment"; Anzahl von DeltaEvents wird verringert, was bei nachfolgenden Schritt besser für die Laufzeit und den Speicher ist und falls man die Records an sich speichert.
\item 2. Änderung der Referenz Sequenz -> um Anzahl JournalEntries zu minimieren (eigentliches Ziel dieser Arbeit um Arbeitsspeicher Ansprüche zu verringern wenn nachfolgend Algorithmen angewendet werden wollen)
\end{itemize}

\subsubsection{Prä-Prozessierung der Records(DeltaEvents)}
\begin{itemize}
\item Sortierung der Events nach Position, Typ und Value
\item Kombinieren von gleichen Events (Update des packed String)
\item Gruppieren der Events zu abhängigen Regionen (Definition?, Grund/Erklärung von sonstigen Schwierigkeiten)
\item jede abhängige Region wird dann unabhängig prozessiert
\end{itemize}

\subsubsection{Prozessierung: Scoring}
\begin{itemize}
\item Kurz nochmal: Was wird optimiert: JournalEntires
\item Score = Anzahl von JournalEntries die addiert oder abgezogen werden würden FALLS das entsprechende Event in die Referenz eingebaut wird
\item Beschreibung für das Scoring von SNP, DEL, INS, SV mit Beispielen
\item Anmerkung: nicht optimal, da verschiedene Kombinationen von Events auf derselben Sequenz nicht beachtet werden. Wäre außerdem zu Laufzeit aufwändig.
\end{itemize}

\subsubsection{Prozessierubg: Veränderung der Referenz}
\begin{itemize}
\item Der beste Score (am meisten negativ) wird in die Referenz eingebaut
\item Alle Sequenzen müssen daraufhin geupdated werden -> wie
\item Unterscheidung zwischen abhängigen und unabhängigen Events
\item Beschreibung der generischen Veränderung von Events mit Beispielen
\item offset beachten
\item Wiederholen des Scorings und der Veränderung der Referenz solange es event mit negativen Score gibt
\end{itemize}

\subsection{Überführung von prozessierten Events in eine DeltaMap}
\begin{itemize}
\item Wozu? -> Anwendbarkeit von Algorithmen
\item Wie .... 
\item Speicherformat/Ausgabeformat
\end{itemize}
\pagebreak
\section{Resultate}
\begin{itemize}
\item auf welchem Server wurde getestet
\item auf welchen Daten wurde getestet, wo kommen sie her und wie wurden sie eventuell prozessiert
\end{itemize}

\subsection{Laufzeit und Speicherplatz}
\begin{itemize}
\item auf Seeding vs. Prozessieren von Events achten
\item Abhängigkeit von Parametern?
\end{itemize}

\subsection{Variant Calling/Seeding}
\begin{itemize}
\item wie viele Seeds bei wie vielen vcf-entries
\item Event verteilung auf dem Chromosom
\item Abhängigkeit von Parametern?
\end{itemize}

\subsection{Komprimierung}
\begin{itemize}
\item Anzahl Records
\item Anzahl JournalEntries
\item Anzahl Bytes im Arbeitsspeicher
\end{itemize}

\section{Diskussion}
\begin{itemize}
\item Auswertung aller Punkte in results.... was fällt auf? wie kann man das Erklären? 
\item Bewertung...
\end{itemize}

\section{Outline}
\begin{itemize}
\item Zusammenfassung der Ergebnisse und Bewertungen
\item Anwendungsmöglichkeiten
\item Verbesserungsmöglichkeiten
\end{itemize}


\end{document}
